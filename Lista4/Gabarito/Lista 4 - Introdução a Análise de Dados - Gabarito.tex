\documentclass[12pt, a4paper]{article}

% Packages and Formatting
\usepackage{../../sub/mystyle_general}
\usepackage{../../sub/mystyle_article}


\title{Lista 4 - Introdução a Análise de Dados \\
	Funções \& Loops}
\author{Guilherme Masuko}
\date{April 2023}
%\affil{}



\definecolor{dkgreen}{rgb}{0,0.6,0}
\definecolor{gray}{rgb}{0.5,0.5,0.5}
\definecolor{mauve}{rgb}{0.58,0,0.82}

\lstset{frame=tb,
	language=R,
	aboveskip=3mm,
	belowskip=3mm,
	showstringspaces=false,
	columns=flexible,
	basicstyle={\small\ttfamily},
	numbers=none,
	numberstyle=\tiny\color{gray},
	keywordstyle=\color{blue},
	commentstyle=\color{dkgreen},
	stringstyle=\color{mauve},
	breaklines=true,
	breakatwhitespace=true,
	tabsize=3
}

\lstset{inputencoding=utf8/latin1}


\begin{document}
	
% Title Page
\clearpage
\maketitle
\thispagestyle{empty}

\textbf{Questão 1}

Crie dois loops. O primeiro deverá printar na tela os valores de 1 à 10 fazendo uso do \texttt{for}. O segundo deverá printar no console os valores de 1 à 5 usando o comando \texttt{while}.



\textbf{Solução}

\lstinputlisting[language=R]{codes/solution1.R}



\textbf{Questão 2}

Crie duas funções, \texttt{ate\_n\_for} e \texttt{ate\_n\_while}, que recebam um parâmetro $n$ e executam os comandos que fizemos na questão anterior. Isso é, a primeira função deverá printar na tela os valores de 1 à $n$ fazendo uso do \texttt{for}. A segunda deverá printar no console os valores de 1 à $n$ usando o comando \texttt{while}.



\textbf{Solução}

\lstinputlisting[language=R]{codes/solution2.R}



\textbf{Questão 3}

Crie um loop que imprima no console a tabuada de 1 à 10.



\textbf{Solução}

\lstinputlisting[language=R]{codes/solution3.R}



\textbf{Questão 4}

\begin{itemize}
	\item[\textbf{a)}] Faça um loop para fazer o somatório de um vetor que vai de 1 à 10.
	
		\begin{align*}
			1+ 2+ 3 + ... + 10
		\end{align*}
	
	
	
	\textbf{Solução}
	
	\lstinputlisting[language=R]{codes/solution4a.R}
	
	
	
	\item[\textbf{b)}] Agora crie uma função que recebe o valor $n$ e retorna o valor da soma.
	
		\begin{align*}
			1+ 2+ 3 + ... + n
		\end{align*}
	
	
	\textbf{Solução}
	
	\lstinputlisting[language=R]{codes/solution4b.R}
	
	
		
\end{itemize}



\textbf{Questão 5}

\begin{itemize}
	\item[\textbf{a)}] Faça um loop que realize o cálculo do fatorial de 10.
	
		\begin{align*}
			10! = 1\cdot 2\cdot 3 \cdot ... \cdot 10
		\end{align*}
	
	
	
	\textbf{Solução}
	
	\lstinputlisting[language=R]{codes/solution5a.R}
	
	

	\item[\textbf{b)}] Agora crie uma função que recebe o valor $n$ e retorne o valor do fatorial.
	
		\begin{align*}
			n! = 1\cdot 2\cdot 3 \cdot ... \cdot n
		\end{align*}
	
	
	
	\textbf{Solução}
	
	\lstinputlisting[language=R]{codes/solution5b.R}
	
	
	
\end{itemize}






\textbf{Questão 6}

\begin{itemize}
	\item[\textbf{a)}] Use o loop \texttt{while} para investigar o número de termos necessários antes que o produto
	
		\begin{align*}
			1\cdot 2\cdot 3\cdot ...
		\end{align*}
	
	alcance mais que um milhão. 
	
	
	
	\textbf{Solução}
	
	\lstinputlisting[language=R]{codes/solution6a.R}
	
	
	
	\item[\textbf{b)}] Agora faça uma função que receba o valor $n$ e retorne o número de termos necessários antes que o produto acima alcance mais que $n$.
	
	
	
	\textbf{Solução}
	
	\lstinputlisting[language=R]{codes/solution6b.R}
	
	
	
\end{itemize}
		


\textbf{Questão 7}

A sequência de Fibonacci\footnote{\url{https://pt.wikipedia.org/wiki/Sequ\%C3\%AAncia\_de\_Fibonacci}} é uma sequência de números inteiros, começando normalmente por 0 e 1, na qual o termo subsequente corresponde à soma dos dois anteriores.

Os números que compõem a sequência de Fibonacci são:

\begin{align*}
	0,1, 1, 2, 3, 5, 8, 13, 21, 34, 55, 89, 144, 233, 377, 610, 987, 1597, 2584, ...
\end{align*}

A sequência é definida recursivamente pela fórmula abaixo.

\begin{align*}
	F_n = F_{n-1} + F_{n-2}
\end{align*}

e valores iniciais $F_1 = 0$ e $F_2 = 1$.

Notação: A notação $(F_n)_{n\in A}$ é usada para denotar a sequência $F$, cujos índices são tomados no conjunto $A$. Quando o conjunto dos índices $A$ está subentendido, normalmente escrevemos $(F_{n})_{{n}}$ ou, simplesmente, $(F_{n})$. Por extenso, escrevemos $(F_{n})_{n}=(F_{1},F_{2},F_{3},\ldots )$. Observamos, ainda, que as notações $\{F_{n}\}_{{n=1}}^{{\infty }}$ e $\{F_{n}\}$ também são encontradas.



\begin{itemize}
	\item[\textbf{a)}] Crie um vetor contendo os primeiros 50 valores da sequência de Fibonacci. Utilize o comando \texttt{for} para adicionar novos elementos ao vetor.
	
	
	
	\textbf{Solução}
	
	\lstinputlisting[language=R]{codes/solution7a.R}
	
	
	
	\item[\textbf{b)}] Crie uma função que recebe como parâmetro o valor $n$. A função deve retornar um vetor contendo os primeiros $n$ valores da sequência de Fibonacci utilizando o programa feito a partir do comando \texttt{for}.
	
	
	
	\textbf{Solução}
	
	\lstinputlisting[language=R]{codes/solution7b.R}
	
	
	
\end{itemize}



\textbf{Questão 8}

\begin{itemize}
	\item[\textbf{a)}] Crie um vetor contendo os primeiros 50 valores da sequência de Fibonacci. Utilize o comando \texttt{while} para adicionar novos elementos ao vetor.
	
	
	
	\textbf{Solução}
	
	\lstinputlisting[language=R]{codes/solution8a.R}
	
	
	
	\item[\textbf{b)}] Crie uma função que recebe como parâmetro o valor $n$. A função deve retornar um vetor contendo os primeiros $n$ valores da sequência de Fibonacci utilizando o programa feito a partir do comando \texttt{while}.
	
	
	
	\textbf{Solução}
	
	\lstinputlisting[language=R]{codes/solution8b.R}
	
	
	
\end{itemize}




	
\end{document}