\documentclass[12pt, a4paper]{article}

% Packages and Formatting
\usepackage{../../sub/mystyle_general}
\usepackage{../../sub/mystyle_article}


\title{Lista 5 - Introdução a Análise de Dados \\
	Dataframes}
\author{Guilherme Masuko}
\date{May 2023}
%\affil{}


\begin{document}
	
% Title Page
\clearpage
\maketitle
\thispagestyle{empty}

Para essa lista, utilizaremos uma base de dados própria do R chamada "Motor Trend Car Road Tests". Utilize os comandos \texttt{data(mtcars)} e \texttt{View(mtcars)} para importar e visualizar os dados respectivamente. Para conhecer melhor essa base de dados (entender o significado de cada coluna, por exemplo), deve-se usar o seguinte comando \texttt{?mtcars}.

\textbf{Questão 1}

Encontre, utilizando código, o carro mais econômico dessa amostra (maior milhas por galão ($mpg$)).



\textbf{Questão 2}

Encontre, utilizando código, o carro menos econômico dessa amostra (menor milhas por galão ($mpg$)).



\textbf{Questão 3}

Dentro da categoria de carros mais potentes (digamos, $hp > 120$), qual é o carro mais econômico?



\textbf{Questão 4}

E dentro da categoria de carros menos potentes (digamos, $hp \leq 120$), qual é o carro menos econômico?



\textbf{Questão 5}

Existe um \textit{trade-off} entre economia e potência de carros? Siga as intruções a seguir para responder essa pergunta.


\begin{itemize}
	\item[\textbf{a)}] Crie uma função que recebe dois vetores de mesmo tamanho e retorne a covariância\footnote{\url{https://en.wikipedia.org/wiki/Covariance}} (amostral) entre esses dois vetores.
	
	A covariância (populacional) entre duas variáveis é obtida através de
	\begin{align*}
		\mathbb{C}(X, Y) = \mathbb{E}\left[ (X - \mathbb{E}[X]) (Y - \mathbb{E}[Y]) \right]
	\end{align*}

	Mas como nossa base não tem todos os carros do mundo, vamos utilizar um estimador para a covariância verdadeiro (populacional), esse estimador é a covariância amostral, obtida através de
	\begin{align*}
		\widehat{\mathbb{C}(X, Y)} = \frac{1}{n-1} \sum_{i=1}^n (X_i - \mu_{X})(Y_i - \mu_{Y})
	\end{align*}
	onde $\mu_{X}$ e $\mu_{Y}$ são as médias amostrais da variáveis $X$ e $Y$, respectivamente.
	
	Obs: O aluno deve utilizar a formula do estimador para criar a função.
	
	\item[\textbf{b)}] Crie uma função que receba dois vetores de mesmo tamanho como parâmetros e retorne a correlação\footnote{\url{https://en.wikipedia.org/wiki/Pearson_correlation_coefficient}} (amostral) entre eles.
	
	A correlação (populacional) entre duas variáveis é obtida através de
	\begin{align*}
		\rho_{X, Y} = \frac{\mathbb{C}(X, Y)}{\sigma_X\cdot \sigma_Y}
	\end{align*}
	onde $\sigma_X$ e $\sigma_Y$ são os devios-padrão populacionais da variáveis $X$ e $Y$, respectivamente.
	
	E seu estimador é a correlação amostral, obtida através de
	\begin{align*}
		\widehat{\rho_{X, Y}} = \frac{\widehat{\mathbb{C}(X, Y)}}{\widehat{\sigma_X}\cdot \widehat{\sigma_Y}}
	\end{align*}
	onde $\widehat{\sigma_X}$ e $\widehat{\sigma_Y}$ são os devios-padrão amostrais da variáveis $X$ e $Y$, respectivamente.
	
	Obs: O aluno deve utilizar a formula do estimador para criar a função.
	
	\item[\textbf{c)}] Calcule a correção entre as duas variáveis economia e potência do carro (representadas por $mpg$ e $hp$, respectivamente) para responder a pergunta principal.
	
\end{itemize}



\textbf{Questão 6}

Crie uma coluna ($wt\_kg$) que contenha o peso de libras ($wt$) convertido em quilogramas. Note que a medida da coluna $wt$ equivale a 1000 libras. A fórmula de conversão é:

\begin{align*}
	\text{peso em kg} = \frac{\text{peso em libras}}{2.2046}
\end{align*}



\textbf{Questão 7}

Qual é o peso médio, em quilogramas, dos carros?



\textbf{Questão 8}

Qual é o peso médio, em quilogramas, dos carros automáticos ($am = 0$)?



\textbf{Questão 9}

Qual é o peso médio, em quilogramas, dos carros manuais ($am = 1$)?



\textbf{Questão 10}

Qual é a correlação entre essas duas variáveis, transmissão ($am$) e peso em quilogramas ($wt\_kg$)? O que isso significa?


\textbf{Probabilidade}

A probabilidade de um evento $A \subset \Omega$ ocorrer é

\begin{align*}
	\mathbb{P}(A) &= \frac{n(A)}{n(\Omega)}\\
	&= \frac{\text{número de elementos do evento}}{\text{número de elementos do espaço amostral}}
\end{align*}

\textbf{Questão 11}

Qual a probabilidade (amostral) de pegarmos (em nossa amostra) um carro manual mais pesado que o peso médio dos carros automáticos?



\textbf{Questão 12}

Qual a probabilidade (amostral) de pegarmos (em nossa amostra) um carro automático mais pesado que o peso médio dos carros manuais?



\textbf{Questão 13}

Qual a probabilidade (amostral) de pegarmos (em nossa amostra) um carro manual mais pesado que o carro automático mais leve?



\textbf{Questão 14}

Qual a probabilidade (amostral) de pegarmos (em nossa amostra) um carro automático mais leve que o carro manual mais pesado?




	
\end{document}